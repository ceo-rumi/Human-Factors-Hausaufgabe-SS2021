\title{Aufgabe 1}

\subsection{}
1. What research questions are investigated in this paper ? 
The first research question regards to the topic if it is not possible to apply standard user interface techniques to fulfil the concept of security. 
The next research questions regards to the topic how the usability of a software user interface can effect security considerations. 
The third research questions if the users (obviously with PGP 5.0) are meeting the expectations regarding the “definition of usability for security”. 
The next research question is where the lack of usability is existing in the used software program. 


\subsection{}
2. What is the motivation behind this research? 
The main motivation in this paper is to show that the users have difficulties while using a security software due to the given usability standards. And what points of usability can be improved or should be improved so that users don’t make mistakes or lead them to unawareness of their user behaviour. 

\subsection{}
3. Compare defnition of usable security from the paper with the ISO definition of
usability from the lecture. What aspects are not explicitly considered by Whitten
and Tygar? Which aspects from the defnition by Whitten and Tygar are missing in
the ISO definition?

The following are the points, which are not explicity considered by Whitten and Tygar: 
•	They don’t mention a specific group of user or goals.
•	They don’t mention any goals, which need to be met. 

The following are the points, which are missing in the ISO definition:
•	If users are reliably aware of the security tasks they need to perform
•	If users are able to figure out how to successfully perform those tasks 
•	If users don’t make dangerous errors 
•	If users are comfortable enough with the interface to continue using it. 

\subsection{}
4. What problematic “properties of security” (Eigenschaften der IT-Sicherheit) should be considered when designing user-friendly security mechanisms? Explain each property in a short sentence and give a an example. 

-	The unmotivated user property: For users, security is not a conscious high priority. They want to use the program to meet their tasks or needs. Example: Saving credit card information’s in a word file for an easier copy-paste-operation .
-	The abstraction property: This property claims that most of the users are not aware of security policies and therefore an optimal interface design should consider this too. Example: No Step-by-Step Tutorials on the interface to see how to encrypt and decrypt an Email and so on. 
-	The lack of feedback property: For security management feedback of the users are very important, but also it depends very strong from wishes of the users. And guessing what the user really wants is difficult. Example: Mostly users don’t have the motivation to give a feedback about the used software, because it takes too long.
-	The barn door property: This property claims that the user interface should make the users aware what security means and what high-cost mistakes could be. Example: Explaining the user how important it is to log out safety compliant. 
-	The weakest link property: At this point users should be careful enough in every security aspect to avoid gaps for attackers. Example: Safe and aware usage of Mobile Apps for online banking.

\subsection{}
5.What usability evaluation methods were used to collect the data? Explain them briefly. What are the advantages and disadvantages of these methods compared to each other? Why did the researchers decide to combine the methods? 
In this paper they used two evaluation methods. The one is called “Cognitive Walkthrough” with some aspects of heuristic evaluation and the second one is called “User Test”. 

The first method can turn out to be difficult because you have to present the user a “real” scenario as much as possible. Therefore it can be said that the focus of the users is not on the security aspect instead the users find something that would be worth additional security etc.. But the advantage of this method is the output. You can see in a detailed form where the user has problems or difficulties.
The second method is an “User Test”. It is a good practice to test the learnability and its specific property of a user test that the previously defined goals will be observed. A disadvantage can be if someone (like experts) familiar with the system or program, he or she is going to miss things that someone who lacks that familiarity would find .
Due to these reasons, it is important to combine these two methods and use them together. An additional reason is also that these methods complete themselves with each other. 

\subsection{}
6. In the paper, the researches list a number of irreversible actions associated with email encryption. List these and describe the resulting consequences for the functionality, availability and usability of email encryption from both the senders and the recipient’s perspective. Can you thing of aspects that go beyond the consequences mentioned in the paper?

-	Accidentally deleting the private key: If the key is not backed up there is no way to get the key again. The sender can do nothing in this situation and the communication cannot be continued.
-	Accidentally publicizing a key: In this case the user has the opportunity to generate a revocation certificate for a key. For a novice user (sender) this opportunity is a complex process, which would make the user desperate. This could lead him to dangerous mistakes. And the recipient probably can’t assume whats going on and could start to mistrust.
-	Accidentally revoking a key: A consequence for the sender is that he could get easily become desperate and he cannot work or finish his tasks the way he wants it. The recipient has to deal with consequences too because he cant not get the needed information to finish his work or tasks. Also if the sender doesn’t inform the recipient about the revoked key, the recipient can possibly encrypt with an useless key.
-	Forgetting the pass phrase: Without the pass phrase the sender cannot encrypt or decrypt the information or message. In this case a useless key will be used by  the recipient. 
-	Failing to back up the key rings: The sender can get difficulties after a system-crash (the key pair would get lost) if there is no back up for the key rings also the recipient. 

\subsection{}
7. The results of the cognitive walkthrough evaluate, among other things, the usability of the visual metaphors. What problems do the researchers  identify related to visual metaphors and what suggestions do the make for improvement?
Problems:
-	With the four buttons the user can see that there are four operation in option. But he cannot distinguish between which operations are for private and public keys. 
-	The Signature metaphor can novice user led to a misunderstanding and misbehaviour of the given operation. 
Suggestions:
-	Building an extension to the metaphors, which are distinguishing between public keys for encryption and private keys for decryption.
-	Using a better icon design can led to a better understanding of the given operation. 

\subsection{}
8. Briefly describe the test scenario for the user test and why the scenario was chosen that way? What primary and secondary tasks did the participants have to perform? 
Test Scenario: User is a volunteer for helping in a political campaign and  he / she should coordinate campaign stuff. This scenario was chosen because it motivates the users to keep a secret. 
The primary task of the volunteer is to send the campaign plan updates to other members of the campaign team with PGP. 
The secondary task is to generate a key pair also to get the public keys of the other team members. Additionally the volunteer should make his own public key available. 


\subsection{}
9. How many participants were involved in the user study? What is known about these participants? How were the participants recruited and what requirements had to be met for participation? 

Number of participants: 12
Information about the participants:
-	Gender , Age , Highest education level, Education or career area
Requirements: no knowledge about public key cryptography 
The participants get recruited through advertising posters on the campus, posts on several local newsgroups and through personal contacts. 

\subsection{}
10.How was the user originally planned ? In what way was the test design adapted later and for which reason?
The actual test plan was separated into two parts. 
-	45 minutes session for doing the mentioned task 
-	45 minutes session in which the test monitor would ask to perform specific tasks

But they changed the session time for doing the task to 90 minutes because the users got difficulties to send a signed and encrypted email. Also there was no breaking up the first session for into getting to the second session. The required specific tasks were set by a fictional campaign manager. 

\subsection{}
11. According to the authors, what design strategies would help to establish more userfriendly security? Do you think these strategies are being implemented today? What problems might arise in implementing these strategies?
The authors propose an conceptual model  that should be simple and smaller. Also they say its necessary to investigate in pragmatic ways of paring down security functionality. These functionalities should be truly necessary and appropriate to a given user group without having less security integrity. After establishing an conceptual model the next step is to communicate it to the user quickly and effectively. 

\subsection{}
12. What is the difference between PGP and S/MIME regarding how trust is established? 
PGPs differ fundamentally in how trust is expressed. In PGP trustful third parties can certify the keys of other users (Web of Trust). On another hand S/MIME has a hierarchical certificate system in which a well-known certification authority (CA) can express trust. 

\subsection{}
13. Many companies use S/MIME to secure employee communications. Regardless of the usability of a concrete implementation, what speaks for and against usage of PGP in a company?
One advantage of PGP is that the algorithm is unbreakable. So your data and files and so on are in secure.
The disadvantages of PGP are:
-	Complexity of use: the encrypting process takes time and effort, which can make it complicated. 
-	Key management: For avoiding incorrect usage of PGP or losing keys etc., user need to fully understand how PGP works. 
-	Lack of anonymity: Messages will be encrypted but not anonymized. The sent messages can be traced back. 
-	Compatibility: Sender and recipient have to use the same version of the software to have a successful communication. 





