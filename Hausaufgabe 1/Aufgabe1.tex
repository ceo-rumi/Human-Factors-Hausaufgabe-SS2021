\subsection{Forschungsziele:}
Der Fokus des Artikels liegt auf den menschlichen Einflüssen von Computer Security im hinblick auf Passwörtern. Das Ziel ist herauszufinden weshalb Sicherheitsrichtlinien von Menschen (in diesem Fall Arbeitnehmern) nicht bzw schlecht umgesetzt werden, welche Gründe dazu führen und welche Gegenmaßnahmen es gibt.

\subsection{Motivation der Forschung:}
Da viele Sicherheitssysteme nicht nutzerfreundlich designt sind scheinen sich viele Anwender nicht an die Sicherheitsmechanismen zu halten. Wenn das Problem der Nutzer verstanden ist, können nutzerfreundliche Systeme zu einer deutlich besseren Sicherheit führen.

\subsection{Befragungsmethoden:}
Es wurde ein Webbasierter Fragebogen genutzt um Daten über das Verhalten von Nutzern im Bezug auf Passwörter zu sammeln. Anschließend wurden semi-strukturierte Interviews geführt um Themen aus dem Fragebogen aufzuarbeiten und Informationen der Teilnehmer zu sammeln.

\subsection{Teilnehmer der Studie:}
Insgesamt gab es 139 Reaktionen auf den Fragebogen. Etwa die Hälfte von Angestellten der Organisation A einer Technologiefirma. Die andere Hälfte von Firmen Weltweit. 
In den Interviews wurden Mitarbeiter der Organisation A (Technologie) und der Organisation B (Baugewerbe) befragt. Man könnte annehmen, dass Angestellte einer Technologiefirma tendenziell besser mit dem Thema Cybersicherheit vertraut sind als andere.

\subsection{Nutzer Strategien:}
Passwort aufschreiben: Nutzer müssen sich viele unterschiedliche Passwörter für unterschiedliche Anwendungen merken und teilweise öfters wechseln. Daraus folgt eine verringerte Sicherheit da Angreifer den Zettel finden könnten. 
Schlechte(einfache) Passwörter: Die Anwender haben viele unterschiedliche Passwörter, die regelmäßig geändert werden müssen. Die Folge sind Passwörter die einfach zu erraten/cracken sind.
Passwörter mit kleinen Veränderungen: Um Security Guidelines bei vielen Passwörtern zu erfüllen werden Passwörter die sich nur durch einzelnen Zahlen unterscheiden verwendet. Die Folge ist noch schlechtere Merkbarkeit der Passwörter was zum aufschreiben der Passwörter führt und wiederum die Sicherheit verringert.

\subsection{Need to know:}
Das need to know Prinzip kommt aus dem Militär und besagt, dass je mehr über ein Sicherheitssystem bekannt ist, desto einfacher es ist das System anzugreifen. Auf die IT Sicherheit angewendet wird den Nutzern also so wenig wie möglich über die genutzten Mechanismen gesagt um das System so vermeintlich Sicherer zu machen. Die Autorinnen sind der Meinung, dass fehlendes Wissen und Aufklärung auf Seite der Anwender der Grund für unsicheres Verhalten sind, und dass diese aufgrund ihrer Unwissenheit die Mechanismen missachten. Die Studie zeigt den Fall auf, dass die Firmen ihre Passwörter von system auf benutzergeneriert umstellen, den Anwendern aber nicht die Sicherheitskriterien für sichere Passwörter nennen. Was wiederum zu unsicheren Passwörtern geführt hat.

\subsection{Sicherheits- Bedrohungsmodelle:}
Nutzer denken Passwort cracking sind Angriffe auf einzelne Persönlichkeiten und schätzen ihre in Platz im System als zu unbedeutend ein. Daraufhin missachten diese Personen oft die Sicherheitsmechanismen. 
Einige verstehen die ID im Authentifikationsprozess als eine andere Art Passwort welche sicher ausgewählt und gemerkt werden muss. Das verdoppelt die zu merkenden Daten was wiederum die Motivation der Angestellten senkt. Hier werden Smartcards oder Biometrische Merkmale als alternative zur ID vorgestellt.
Die Folgen eines Security Breaches wird von einigen Nutzern bezüglich der vermeintlich unwichtigen Daten als nicht bedenklich eingeschätzt. Die Folge sind schlechte sicherung der Zugänge oder Daten.

\subsection{Schlussfolgerung der Ergebnisse:}
Die Sichereheitsfeatures sind nicht mit nach den Möglichkeiten des Benutzers Designt. Viele Mechanismen fordern einen großen Overhead an Arbeit die von vielen Arbeitnehmern nicht geleistet werden kann.
Schlechte Kommunikation der Security Abteilung bezüglich der Mechanismen und deren Gründe sind ein Grund für den Motivationsverlust der Arbeiter.
Sicherheits Workshops könnten die Menschen für das Thema IT Sicherheit sensibilisieren.

\subsection{Ergebnisse in der Heutigen Zeit:}
Viele Menschen haben immer noch wenig Berührungspunkte mit der digitalen Welt und sind dementsprechend unvorsichtig mit ihren Daten und Passwörtern. Themen wie social engineering sind durch social media immer populärer geworden. Oft müssen Angestellte sich immer noch viele Passwörter merken, auch wenn die Verbreitung von single sign on, Smart Cards, Two Factor Authentication und Passwort Managern stetig zunimmt. Viele (größere) Firmen bieten Workshops an, die das Thema IT Sicherheit verständlicher machen sollen.

\subsection{Gefahren durch Forschung:}
Die gesammelten Daten sollten nicht direkt mit den Angestellten in Verbindung gebracht werden. Ein eventuell schlechter Umgang mit dem Thema Sicherheit könnte für die betroffenen personelle Konsequenzen nach sich ziehen und eventuell zu Jobverlust, Mobbing oder Diskriminierung am Arbeitsplatz führen. Die gesammelten Daten könnten in den Händen von Kriminellen ein Sicherheitsrisiko für die Firmen darstellen und Optionen für Human Engineering oder anderes Hacking eröffnen. 
Gegenmaßnahmen:
Ein annonymisieren bzw pseudonymisieren der Daten von Mitarbeitern und Unternehmen sollte von den Studienerstellern sorgfältig durchgeführt werden. Die Daten sicher speichern und den Zugriff einschränken. Keine Rückschlüsse auf Personen durch angegebene Daten zulassen.


