\title{Aufgabe 3}

\subsection{}
Das „c“ in der Tabelle steht für den Kostenfaktor.
Kostenfaktor bedeutet hierbei der Rechenaufwand, welcher entsteht wenn man einen Hashwert für ein Passwort berechnen will. 

\subsection{}
Für alle drei Spalten kann das Wachstumsverhältnis exponentiell beschrieben werden. 

\subsection{}
Der SHA-256 ist dazu ausgelegen sehr schnell zu sein. Man möchte hier keine Verzögerungen haben, wenn man zum Beispiel eine Signatur validieren möchte.
Jedoch ist bycrypt hierbei ein Passwort-Hash, welcher eine gewisse Schlüsselstärke für den zu verschlüsselnden Inhalt anbietet. Das geschieht, indem die Kalkulation verlangsamt wird. Die Angreifer müssen in diesem Fall mehr Ressourcen finden um den Inhalt zu finden durch brute forcing usw. 
Deswegen eignet sich bycrypt für die Verschlüsselung von Passwörter.
